\PassOptionsToPackage{unicode=true}{hyperref} % options for packages loaded elsewhere
\PassOptionsToPackage{hyphens}{url}
%
\documentclass[]{article}
\usepackage{lmodern}
\usepackage{amssymb,amsmath}
\usepackage{ifxetex,ifluatex}
\usepackage{fixltx2e} % provides \textsubscript
\ifnum 0\ifxetex 1\fi\ifluatex 1\fi=0 % if pdftex
  \usepackage[T1]{fontenc}
  \usepackage[utf8]{inputenc}
  \usepackage{textcomp} % provides euro and other symbols
\else % if luatex or xelatex
  \usepackage{unicode-math}
  \defaultfontfeatures{Ligatures=TeX,Scale=MatchLowercase}
\fi
% use upquote if available, for straight quotes in verbatim environments
\IfFileExists{upquote.sty}{\usepackage{upquote}}{}
% use microtype if available
\IfFileExists{microtype.sty}{%
\usepackage[]{microtype}
\UseMicrotypeSet[protrusion]{basicmath} % disable protrusion for tt fonts
}{}
\IfFileExists{parskip.sty}{%
\usepackage{parskip}
}{% else
\setlength{\parindent}{0pt}
\setlength{\parskip}{6pt plus 2pt minus 1pt}
}
\usepackage{hyperref}
\hypersetup{
            pdftitle={Ecology Workshop: Project Proposal},
            pdfauthor={Tommy Shannon},
            pdfborder={0 0 0},
            breaklinks=true}
\urlstyle{same}  % don't use monospace font for urls
\usepackage[margin=1in]{geometry}
\usepackage{graphicx,grffile}
\makeatletter
\def\maxwidth{\ifdim\Gin@nat@width>\linewidth\linewidth\else\Gin@nat@width\fi}
\def\maxheight{\ifdim\Gin@nat@height>\textheight\textheight\else\Gin@nat@height\fi}
\makeatother
% Scale images if necessary, so that they will not overflow the page
% margins by default, and it is still possible to overwrite the defaults
% using explicit options in \includegraphics[width, height, ...]{}
\setkeys{Gin}{width=\maxwidth,height=\maxheight,keepaspectratio}
\setlength{\emergencystretch}{3em}  % prevent overfull lines
\providecommand{\tightlist}{%
  \setlength{\itemsep}{0pt}\setlength{\parskip}{0pt}}
\setcounter{secnumdepth}{0}
% Redefines (sub)paragraphs to behave more like sections
\ifx\paragraph\undefined\else
\let\oldparagraph\paragraph
\renewcommand{\paragraph}[1]{\oldparagraph{#1}\mbox{}}
\fi
\ifx\subparagraph\undefined\else
\let\oldsubparagraph\subparagraph
\renewcommand{\subparagraph}[1]{\oldsubparagraph{#1}\mbox{}}
\fi

% set default figure placement to htbp
\makeatletter
\def\fps@figure{htbp}
\makeatother


\title{Ecology Workshop: Project Proposal}
\author{Tommy Shannon}
\date{1/10/2020}

\begin{document}
\maketitle

I intend to analyze a periphyton dataset from the Everglades CERP sites
to understand synchronicity of change in species assemblage with respect
to geographic distance over a 13-year period; for sites that exhibit
synchronous patterns, I plan to look for explanatory variables such as
distance between sites and environmental drivers including phosphorus
and chloride. Ultimately, my hope is to use a similar dataset to test
for the presence of chemical intrusion such as saltwater and fertilizer
into the same system, but for the purposes of this project my goal is
more simply to understand the geotemporal patterns of algal community
dynamics and correlate them with similar chemical changes.

I hypothesize that synchronous algal assemblage dynamics will be found
among sites, but that intra-site autocorrelation may not be apparent. I
additionally hypothesize that those synchrony dynamics will negatively
correlate with both geographic and environmental distance between sites,
but that those two distance measurements will not strongly correlate
with each other.

The dataset I'm working with is one of species-level diatom enumeration
alongside data on chemical, physical, and biological properties at each
of the sites sampled. The data is sampled from 150 sites within the
Florida Everglades CERP area, collected from 2006-2018 with funding from
the Army Corps of Engineers. Sites were sampled once per year at a
random specific location within a larger 500m\^{}2 quadrant. I have some
concern that the temporal resolution may be too coarse to see assemblage
patterns which may be present, but the once-yearly sampling will allow
me to possibly see longer-scale patterns and trends over the course of
the system's water restoration process.

Part of my hope for this class is to learn how to best analyze and
perform statistical analysis on geotemporally rich datasets like this
one. My plan is to perform matrix regression analyses for beta diversity
vs.~geographic or environmental distance, but I am eager to better
understand what this entails and find out if there are better ways of
approaching this data. The course modules on timeseries analysis and
species distribution modeling seem incredibly relevant to my dataset,
and so I plan to make the most of them.

\end{document}
